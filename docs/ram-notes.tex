% Options for packages loaded elsewhere
\PassOptionsToPackage{unicode}{hyperref}
\PassOptionsToPackage{hyphens}{url}
%
\documentclass[
]{book}
\usepackage{lmodern}
\usepackage{amsmath}
\usepackage{ifxetex,ifluatex}
\ifnum 0\ifxetex 1\fi\ifluatex 1\fi=0 % if pdftex
  \usepackage[T1]{fontenc}
  \usepackage[utf8]{inputenc}
  \usepackage{textcomp} % provide euro and other symbols
  \usepackage{amssymb}
\else % if luatex or xetex
  \usepackage{unicode-math}
  \defaultfontfeatures{Scale=MatchLowercase}
  \defaultfontfeatures[\rmfamily]{Ligatures=TeX,Scale=1}
\fi
% Use upquote if available, for straight quotes in verbatim environments
\IfFileExists{upquote.sty}{\usepackage{upquote}}{}
\IfFileExists{microtype.sty}{% use microtype if available
  \usepackage[]{microtype}
  \UseMicrotypeSet[protrusion]{basicmath} % disable protrusion for tt fonts
}{}
\makeatletter
\@ifundefined{KOMAClassName}{% if non-KOMA class
  \IfFileExists{parskip.sty}{%
    \usepackage{parskip}
  }{% else
    \setlength{\parindent}{0pt}
    \setlength{\parskip}{6pt plus 2pt minus 1pt}}
}{% if KOMA class
  \KOMAoptions{parskip=half}}
\makeatother
\usepackage{xcolor}
\IfFileExists{xurl.sty}{\usepackage{xurl}}{} % add URL line breaks if available
\IfFileExists{bookmark.sty}{\usepackage{bookmark}}{\usepackage{hyperref}}
\hypersetup{
  pdftitle={Reticular Action Model (RAM) Notation},
  pdfauthor={Ivan Jacob Agaloos Pesigan},
  hidelinks,
  pdfcreator={LaTeX via pandoc}}
\urlstyle{same} % disable monospaced font for URLs
\usepackage{color}
\usepackage{fancyvrb}
\newcommand{\VerbBar}{|}
\newcommand{\VERB}{\Verb[commandchars=\\\{\}]}
\DefineVerbatimEnvironment{Highlighting}{Verbatim}{commandchars=\\\{\}}
% Add ',fontsize=\small' for more characters per line
\usepackage{framed}
\definecolor{shadecolor}{RGB}{248,248,248}
\newenvironment{Shaded}{\begin{snugshade}}{\end{snugshade}}
\newcommand{\AlertTok}[1]{\textcolor[rgb]{0.94,0.16,0.16}{#1}}
\newcommand{\AnnotationTok}[1]{\textcolor[rgb]{0.56,0.35,0.01}{\textbf{\textit{#1}}}}
\newcommand{\AttributeTok}[1]{\textcolor[rgb]{0.77,0.63,0.00}{#1}}
\newcommand{\BaseNTok}[1]{\textcolor[rgb]{0.00,0.00,0.81}{#1}}
\newcommand{\BuiltInTok}[1]{#1}
\newcommand{\CharTok}[1]{\textcolor[rgb]{0.31,0.60,0.02}{#1}}
\newcommand{\CommentTok}[1]{\textcolor[rgb]{0.56,0.35,0.01}{\textit{#1}}}
\newcommand{\CommentVarTok}[1]{\textcolor[rgb]{0.56,0.35,0.01}{\textbf{\textit{#1}}}}
\newcommand{\ConstantTok}[1]{\textcolor[rgb]{0.00,0.00,0.00}{#1}}
\newcommand{\ControlFlowTok}[1]{\textcolor[rgb]{0.13,0.29,0.53}{\textbf{#1}}}
\newcommand{\DataTypeTok}[1]{\textcolor[rgb]{0.13,0.29,0.53}{#1}}
\newcommand{\DecValTok}[1]{\textcolor[rgb]{0.00,0.00,0.81}{#1}}
\newcommand{\DocumentationTok}[1]{\textcolor[rgb]{0.56,0.35,0.01}{\textbf{\textit{#1}}}}
\newcommand{\ErrorTok}[1]{\textcolor[rgb]{0.64,0.00,0.00}{\textbf{#1}}}
\newcommand{\ExtensionTok}[1]{#1}
\newcommand{\FloatTok}[1]{\textcolor[rgb]{0.00,0.00,0.81}{#1}}
\newcommand{\FunctionTok}[1]{\textcolor[rgb]{0.00,0.00,0.00}{#1}}
\newcommand{\ImportTok}[1]{#1}
\newcommand{\InformationTok}[1]{\textcolor[rgb]{0.56,0.35,0.01}{\textbf{\textit{#1}}}}
\newcommand{\KeywordTok}[1]{\textcolor[rgb]{0.13,0.29,0.53}{\textbf{#1}}}
\newcommand{\NormalTok}[1]{#1}
\newcommand{\OperatorTok}[1]{\textcolor[rgb]{0.81,0.36,0.00}{\textbf{#1}}}
\newcommand{\OtherTok}[1]{\textcolor[rgb]{0.56,0.35,0.01}{#1}}
\newcommand{\PreprocessorTok}[1]{\textcolor[rgb]{0.56,0.35,0.01}{\textit{#1}}}
\newcommand{\RegionMarkerTok}[1]{#1}
\newcommand{\SpecialCharTok}[1]{\textcolor[rgb]{0.00,0.00,0.00}{#1}}
\newcommand{\SpecialStringTok}[1]{\textcolor[rgb]{0.31,0.60,0.02}{#1}}
\newcommand{\StringTok}[1]{\textcolor[rgb]{0.31,0.60,0.02}{#1}}
\newcommand{\VariableTok}[1]{\textcolor[rgb]{0.00,0.00,0.00}{#1}}
\newcommand{\VerbatimStringTok}[1]{\textcolor[rgb]{0.31,0.60,0.02}{#1}}
\newcommand{\WarningTok}[1]{\textcolor[rgb]{0.56,0.35,0.01}{\textbf{\textit{#1}}}}
\usepackage{longtable,booktabs}
\usepackage{calc} % for calculating minipage widths
% Correct order of tables after \paragraph or \subparagraph
\usepackage{etoolbox}
\makeatletter
\patchcmd\longtable{\par}{\if@noskipsec\mbox{}\fi\par}{}{}
\makeatother
% Allow footnotes in longtable head/foot
\IfFileExists{footnotehyper.sty}{\usepackage{footnotehyper}}{\usepackage{footnote}}
\makesavenoteenv{longtable}
\usepackage{graphicx}
\makeatletter
\def\maxwidth{\ifdim\Gin@nat@width>\linewidth\linewidth\else\Gin@nat@width\fi}
\def\maxheight{\ifdim\Gin@nat@height>\textheight\textheight\else\Gin@nat@height\fi}
\makeatother
% Scale images if necessary, so that they will not overflow the page
% margins by default, and it is still possible to overwrite the defaults
% using explicit options in \includegraphics[width, height, ...]{}
\setkeys{Gin}{width=\maxwidth,height=\maxheight,keepaspectratio}
% Set default figure placement to htbp
\makeatletter
\def\fps@figure{htbp}
\makeatother
\setlength{\emergencystretch}{3em} % prevent overfull lines
\providecommand{\tightlist}{%
  \setlength{\itemsep}{0pt}\setlength{\parskip}{0pt}}
\setcounter{secnumdepth}{5}
\usepackage{booktabs}
\usepackage{amsthm}
\makeatletter
\def\thm@space@setup{%
  \thm@preskip=8pt plus 2pt minus 4pt
  \thm@postskip=\thm@preskip
}
\makeatother
\ifluatex
  \usepackage{selnolig}  % disable illegal ligatures
\fi
\usepackage[]{natbib}
\bibliographystyle{apalike}

\title{Reticular Action Model (RAM) Notation}
\author{Ivan Jacob Agaloos Pesigan}
\date{2021-01-08}

\begin{document}
\maketitle

{
\setcounter{tocdepth}{1}
\tableofcontents
}
\hypertarget{description}{%
\chapter{Description}\label{description}}

This is a collection of my personal notes on the Reticular Action Model (RAM) notation
that accompanies the \texttt{ram} package.
You can install the released version of \texttt{ram} from \href{https://github.com/jeksterslab/ram}{GitHub} with:

\begin{Shaded}
\begin{Highlighting}[]
\NormalTok{remotes}\SpecialCharTok{::}\FunctionTok{install\_github}\NormalTok{(}\StringTok{"jeksterslab/ram"}\NormalTok{)}
\end{Highlighting}
\end{Shaded}

These notes are based on the following resources:

\begin{itemize}
\tightlist
\item
  \citet{Boker-2005}
\item
  \citet{McArdle-1984}
\item
  \citet{McArdle-2005}
\end{itemize}

See \href{https://jeksterslab.github.io/ram_notes/index.html}{GitHub Pages}
for the html deployment.

\hypertarget{ram-matrix-notation}{%
\chapter{Reticular Action Model (RAM) Matrix Notation}\label{ram-matrix-notation}}

\begin{equation}
  \mathbf{v}
  =
  \mathbf{A} \mathbf{v} + \mathbf{u}
\end{equation}

\noindent where

\begin{itemize}
\tightlist
\item
  \(\mathbf{v}\) is a \(t \times 1\) vector of random variables
\item
  \(u_i\) represent the residual of \(v_i\)
\item
  \(\mathbf{A}\) is a \(t \times t\) matrix of \emph{directed} or \emph{asymmetric} relationship
  from column variable \(v_j\) to row variable \(v_i\)

  \begin{itemize}
  \tightlist
  \item
    regression of each of the \(t\) variables on the other \(t - 1\) variables
  \item
    diagonal \(a_{i,i}\) is zero
  \item
    if all regression coefficients on other variables are zero,
    then the variable \(v_i\) is considered the same as its own residual \(u_i\)
  \end{itemize}
\end{itemize}

\begin{equation}
  \boldsymbol{\Omega}
  =
  \mathbb{E}
  \left(
    \mathbf{u} \mathbf{u}^{\prime}
  \right)
\end{equation}

\noindent where

\begin{itemize}
\tightlist
\item
  \(\boldsymbol{\Omega}\) is a \(t \times t\) matrix of \emph{undirected} or \emph{symmetric} relationship
\end{itemize}

\begin{equation}
  \boldsymbol{\Sigma} \left( \boldsymbol{\theta} \right)
  =
  \left( \mathbf{I} - \mathbf{A} \right)^{-1}
  \boldsymbol{\Omega}
  \left[ \left( \mathbf{I} - \mathbf{A} \right)^{-1} \right]^{\mathsf{T}}
\end{equation}

\begin{itemize}
\tightlist
\item
  \(\boldsymbol{\Sigma} \left( \boldsymbol{\theta} \right)\) is a \(t \times t\) symmetric matrix
  of associations between \(v_i\) and \(v_j\)
\end{itemize}

\begin{equation}
  \mathbf{v}^{\mathsf{T}}
  =
  \left[
    \mathbf{m}, \mathbf{l}
  \right]^{\mathsf{T}}
\end{equation}

\noindent where

\begin{itemize}
\tightlist
\item
  \(\mathbf{m}\) are observed or manifest variables of \(j\) components
\item
  \(\mathbf{l}\) are observed or manifest variables of \(k\) components
\item
  \(t = j + k\)
\end{itemize}

\begin{equation}
  \mathbf{F}
  =
  \left[
    \mathbf{I}_{j} \colon \mathbf{O}_{j \times k}
  \right]
\end{equation}

\begin{itemize}
\tightlist
\item
  the \(\mathbf{F}\) matrix acts as a \emph{filter} to select the manifest variables
  out of the full set of manifest and latent variables
\end{itemize}

\hypertarget{model-implied-matrices}{%
\section{Model-Implied Matrices}\label{model-implied-matrices}}

The model-implied mean vector
\(\boldsymbol{\mu} \left( \boldsymbol{\theta} \right)\)
as a function of Reticular Action Model (RAM) matrices
is given by

\begin{equation}
  \boldsymbol{\mu} \left( \boldsymbol{\theta} \right)
  =
  \mathbf{F}
  \left( \mathbf{I} - \mathbf{A} \right)^{-1}
  \mathbf{m} .
\end{equation}

\noindent The \texttt{ram::mutheta()} function can be used
to derive the model-implied mean vector.

The model-implied variance-covariance matrix
\(\boldsymbol{\Sigma} \left( \boldsymbol{\theta} \right)\)
as a function of Reticular Action Model (RAM) matrices
is given by

\begin{equation}
  \boldsymbol{\Sigma} \left( \boldsymbol{\theta} \right)
  =
  \mathbf{F}
  \left( \mathbf{I} - \mathbf{A} \right)^{-1}
  \boldsymbol{\Omega}
  \left[ \left( \mathbf{I} - \mathbf{A} \right)^{-1} \right]^{\mathsf{T}}
  \mathbf{F}^{\mathsf{T}} .
\end{equation}

\noindent The \texttt{ram::Sigmatheta()} function can be used
to derive the model-implied variance-covariance matrix.

\hypertarget{parameters}{%
\section{Parameters}\label{parameters}}

\hypertarget{mean-structure}{%
\subsection{Mean Structure}\label{mean-structure}}

\begin{equation}
  \mathbf{m}
  =
  \left[
    \mathbf{F}
    \left( \mathbf{I} - \mathbf{A} \right)^{-1}
  \right]^{-1}
  \boldsymbol{\mu} \left( \boldsymbol{\theta} \right)
\end{equation}

\noindent The \texttt{ram::m()} function can be used
to derive the mean structure vector.

\hypertarget{covariance-structure}{%
\subsection{Covariance Structure}\label{covariance-structure}}

\begin{equation}
  \boldsymbol{\Omega}
  =
  \left( \mathbf{I} - \mathbf{A} \right)
  \boldsymbol{\Sigma} \left( \boldsymbol{\theta} \right)
  \left( \mathbf{I} - \mathbf{A} \right)^{\mathsf{T}}
\end{equation}

\noindent The \texttt{ram::Omega()} function can be used
to derive the \emph{symmetric} matrix \(\boldsymbol{\Omega}\).

TODO: Figure out how to isolate the A matrix

\hypertarget{simple-regression}{%
\chapter{Simple Regression}\label{simple-regression}}

Let \(v_1\), \(v_2\), and \(v_3\) be random variables whose associations are given by the regression equation

\begin{equation}
  \begin{split}
    v_1
    &=
    m_1 + a_{1, 2} v_2 + v_3 \\
    &=
    -3.951208 + 1.269259 \cdot v_2 + v_3 .
  \end{split}
\end{equation}

\noindent \(v_1\) and \(v_2\) are observed variables and
\(v_3\) is a stochastic error term which is normally distributed around zero
with constant variance across values of \(v_2\)

\begin{equation}
  v_3 
  \sim 
  \mathcal{N} \left( m_3 = 0, \omega_{3, 3} = 47.659854 \right) .
\end{equation}

\noindent \(v_2\) has a mean of \(m_2 = 13.038328\) and a variance of \(\omega_{2, 2} = 7.151261\).

\hypertarget{expectations}{%
\subsection{Expectations}\label{expectations}}

\begin{equation}
  \begin{split}
    \mathbb{E}
    \left(
      v_3
    \right)
    &=
    m_3 \\
    &=
    0
  \end{split}
\end{equation}

\begin{equation}
  \begin{split}
    \mathbb{E}
    \left(
      v_2
    \right)
    &=
    m_2 \\
    &=
  13.038328
  \end{split}
\end{equation}

\begin{equation}
  \begin{split}
    \mathbb{E}
    \left(
      v_1
    \right)
    &=
    \mathbb{E}
    \left(
      m_1 + a_{1, 2} v_2 + v_3
    \right) \\
    &=
    \mathbb{E}
    \left(
      m_1
    \right)
    +
    \mathbb{E}
    \left(
      a_{1, 2} v_2
    \right)
    +
    \mathbb{E}
    \left(
      v_3
    \right) \\
    &=
    m_1
    +
    a_{1, 2}
    \mathbb{E}
    \left(
      v_2
    \right)
    +
    0 \\
    &=
    m_1
    +
    a_{1, 2}
    m_2 \\
    &=
    -3.951208
    +
    1.269259
    \times
    13.038328 \\
    &=
    12.5978072
  \end{split}
\end{equation}

\begin{equation}
  \begin{split}
    \mathbb{E}
    \left(
      \begin{bmatrix}
        v_1 \\
        v_2 \\
        v_3
      \end{bmatrix}
    \right)
    &=
    \begin{bmatrix}
      m_1 + a_{1, 2} m_2 \\
      m_2 \\
      m_3
    \end{bmatrix} \\
    &=
    \begin{bmatrix}
      12.5978072 \\
      13.038328 \\
      0
    \end{bmatrix}
  \end{split}
\end{equation}

\begin{equation}
  \begin{split}
    \mathrm{Cov}
    \left(
      v_3,
      v_3
    \right)
    &=
    \mathrm{Var}
    \left(
      v_3
    \right) \\
    &=
    \omega_{3, 3} \\
    &=
    47.659854
  \end{split}
\end{equation}

\begin{equation}
  \begin{split}
    \mathrm{Cov}
    \left(
      v_1,
      v_3
    \right)
    &=
    \mathrm{Cov}
    \left(
      a_{1, 2} v_2 + v_3,
      v_3 
    \right) \\
    &=
    \mathrm{Cov}
    \left(
      a_{1, 2} v_2, v_3
    \right)
    +
    \mathrm{Cov}
    \left(
      v_3, v_3
    \right) \\
    &=
    a_{1, 2}^{2}
    \mathrm{Cov}
    \left(
      v_2, v_3
    \right)
    +
    \mathrm{Var}
    \left(
      v_3
    \right) \\
    &=
    a_{1, 2}^{2}
    \cdot
    0
    +
    \omega_{3, 3} \\
    &=
    0
    +
    \omega_{3, 3} \\
    &=
    \omega_{3, 3} \\
    &=
    47.659854
  \end{split}
\end{equation}

\begin{equation}
  \mathrm{Cov}
  \left(
    v_2,
    v_3
  \right)
  =
  0
\end{equation}

\begin{equation}
  \begin{split}
    \mathrm{Cov} \left( v_1, v_1 \right)
    &=
    \mathrm{Cov}
    \left(
      a_{1, 2} v_2 + v_3,
      a_{1, 2} v_2 + v_3
    \right) \\
    &=
    \mathrm{Cov} \left( a_{1, 2} v_2, a_{1, 2} v_2 \right)
    +
    \mathrm{Cov} \left( a_{1, 2} v_2, v_3 \right)
    +
    \mathrm{Cov} \left( a_{1, 2} v_2, v_3 \right)
    +
    \mathrm{Cov} \left( v_3, v_3 \right) \\
    &=
    a_{1, 2}^{2} \mathrm{Cov} \left( v_2, v_2 \right)
    +
    a_{1, 2} \mathrm{Cov} \left( v_2, v_3 \right)
    +
    a_{1, 2} \mathrm{Cov} \left( v_2, v_3 \right)
    +
    \mathrm{Var} \left( v_3 \right) \\
    &=
    a_{1, 2}^{2} \mathrm{Var} \left( v_2 \right)
    +
    a_{1, 2} \cdot 0
    +
    a_{1, 2} \cdot 0
    +
    \omega_{3, 3} \\
    &=
    a_{1, 2}^{2} \mathrm{Var} \left( v_2 \right)
    +
    0
    +
    0
    +
    \omega_{3, 3} \\
    &=
    a_{1, 2}^{2} \omega_{2, 2} + \omega_{3, 3} \\
    &=
    1.269259^{2} \times 7.151261 + 47.659854 \\
    &=
    59.1806671
  \end{split}
\end{equation}

\begin{equation}
  \begin{split}
    \mathrm{Cov} \left( v_2, v_1 \right)
    &=
    \mathrm{Cov} \left( v_2, a_{1, 2} v_2 + v_3 \right) \\
    &=
    \mathrm{Cov} \left( v_2, a_{1, 2} v_2 \right)
    +
    \mathrm{Cov} \left( v_2, v_3 \right) \\
    &=
    a_{1, 2} \mathrm{Cov} \left( v_2, v_2 \right)
    +
    0 \\
    &=
    a_{1, 2} \mathrm{Var} \left( v_2 \right) \\
    &=
    a_{1, 2} \omega_{2, 2} \\
    &=
    1.269259 \times 7.151261 \\ 
    &=
    9.0768024 \\ 
  \end{split}
\end{equation}

\begin{equation}
  \begin{split}
    \mathrm{Cov} \left( v_2, v_2 \right)
    &=
    \mathrm{Var} \left( v_2 \right) \\
    &=
    \omega_{2, 2} \\
    &=
    7.151261
  \end{split}
\end{equation}

\begin{equation}
  \begin{split}
    \mathrm{Cov}
    \left(
      \begin{bmatrix}
        v_1 \\
        v_2 \\
        v_3
      \end{bmatrix}
    \right)
    &=
    \begin{bmatrix}
      a_{1, 2}^{2} \omega_{2, 2} + \omega_{3, 3} & a_{1, 2} \omega_{2, 2} & \omega_{3, 3} \\
      a_{1, 2} \omega_{2, 2} & \omega_{2, 2} & 0 \\
      \omega_{3, 3} & 0 & \omega_{3, 3}
    \end{bmatrix} \\
    &=
    \begin{bmatrix}
      59.1806671 & 9.0768024 & 47.659854 \\
      9.0768024 & 7.151261 & 0 \\
      47.659854 & 0 & 47.659854
    \end{bmatrix}
  \end{split}
\end{equation}

Below are two ways of specifying this model.
The first specification includes the error term \(v_3\) as a latent variable.
The second specification only includes the observed variables.

\hypertarget{specification-1---includes-error-term-as-a-latent-variable}{%
\section{Specification 1 - Includes Error Term as a Latent Variable}\label{specification-1---includes-error-term-as-a-latent-variable}}

\hypertarget{matrix-notation}{%
\subsection{Matrix Notation}\label{matrix-notation}}

\begin{equation}
  \mathbf{v}
  =
  \begin{bmatrix}
    v_1 \\
    v_2 \\
    v_3
  \end{bmatrix}
\end{equation}

\begin{equation}
  \begin{split}
    \mathbf{A}
    &=
    \begin{bmatrix}
      0 & a_{1, 2} & 1 \\
      0 & 0        & 0 \\
      0 & 0        & 0
    \end{bmatrix} \\
    &=
    \begin{bmatrix}
      0 & 1.269259 & 1 \\
      0 & 0 & 0 \\
      0 & 0 & 0
    \end{bmatrix}
  \end{split}
\end{equation}

\begin{equation}
  \begin{split}
    \boldsymbol{\Omega}
    &=
    \begin{bmatrix}
      0 & 0             & 0 \\
      0 & \omega_{2, 2} & 0 \\
      0 & 0             & \omega_{3, 3}
    \end{bmatrix} \\
    &=
    \begin{bmatrix}
      0 & 0 & 0 \\
      0 & 7.151261 & 0 \\
      0 & 0 & 47.659854
    \end{bmatrix}
  \end{split}
\end{equation}

\begin{equation}
  \begin{split}
    \mathbf{m}
    &=
    \begin{bmatrix}
      m_1 \\
      m_2 \\
      m_3
    \end{bmatrix} \\
    &=
    \begin{bmatrix}
      -3.951208 \\
      13.038328 \\
      0
    \end{bmatrix}
  \end{split}
\end{equation}

To filter the observed variables, use the following filter matrix

\begin{equation}
  \mathbf{F}
  =
  \begin{bmatrix}
    1 & 0 & 0 \\
    0 & 1 & 0
  \end{bmatrix} .
\end{equation}

To include all variables, use the following filter matrix

\begin{equation}
  \mathbf{F}
  =
  \begin{bmatrix}
    1 & 0 & 0 \\
    0 & 1 & 0 \\
    0 & 0 & 1
  \end{bmatrix} .
\end{equation}

\begin{figure}
\centering
\includegraphics{ram-notes_files/figure-latex/regression-01-1.pdf}
\caption{\label{fig:regression-01}The Simple Linear Regression Model (with \(v_3\))}
\end{figure}

\hypertarget{using-the-ram-package}{%
\subsubsection{\texorpdfstring{Using the \texttt{ram()} Package}{Using the ram() Package}}\label{using-the-ram-package}}

\begin{Shaded}
\begin{Highlighting}[]
\NormalTok{knitr}\SpecialCharTok{::}\FunctionTok{kable}\NormalTok{(}
\NormalTok{  ram}\SpecialCharTok{::}\FunctionTok{mutheta}\NormalTok{(}
\NormalTok{    m,}
    \AttributeTok{A =}\NormalTok{ A,}
    \AttributeTok{filter =}\NormalTok{ filter}
\NormalTok{  ),}
  \AttributeTok{col.names =} \StringTok{"$}\SpecialCharTok{\textbackslash{}\textbackslash{}}\StringTok{boldsymbol\{}\SpecialCharTok{\textbackslash{}\textbackslash{}}\StringTok{mu\}$"}\NormalTok{,}
  \AttributeTok{caption =} \StringTok{"$}\SpecialCharTok{\textbackslash{}\textbackslash{}}\StringTok{boldsymbol\{}\SpecialCharTok{\textbackslash{}\textbackslash{}}\StringTok{mu\} }\SpecialCharTok{\textbackslash{}\textbackslash{}}\StringTok{left( }\SpecialCharTok{\textbackslash{}\textbackslash{}}\StringTok{boldsymbol\{}\SpecialCharTok{\textbackslash{}\textbackslash{}}\StringTok{theta\} }\SpecialCharTok{\textbackslash{}\textbackslash{}}\StringTok{right)$"}\NormalTok{,}
  \AttributeTok{escape =} \ConstantTok{FALSE}
\NormalTok{)}
\end{Highlighting}
\end{Shaded}

\begin{table}

\caption{\label{tab:unnamed-chunk-3}$\boldsymbol{\mu} \left( \boldsymbol{\theta} \right)$}
\centering
\begin{tabular}[t]{l|r}
\hline
  & $\boldsymbol{\mu}$\\
\hline
$v_1$ & 12.59781\\
\hline
$v_2$ & 13.03833\\
\hline
$v_3$ & 0.00000\\
\hline
\end{tabular}
\end{table}

\begin{Shaded}
\begin{Highlighting}[]
\NormalTok{knitr}\SpecialCharTok{::}\FunctionTok{kable}\NormalTok{(}
\NormalTok{  ram}\SpecialCharTok{::}\FunctionTok{Sigmatheta}\NormalTok{(}
    \AttributeTok{A =}\NormalTok{ A,}
    \AttributeTok{Omega =}\NormalTok{ Omega,}
    \AttributeTok{filter =}\NormalTok{ filter}
\NormalTok{  ),}
  \AttributeTok{caption =} \StringTok{"$}\SpecialCharTok{\textbackslash{}\textbackslash{}}\StringTok{boldsymbol\{}\SpecialCharTok{\textbackslash{}\textbackslash{}}\StringTok{Sigma\} }\SpecialCharTok{\textbackslash{}\textbackslash{}}\StringTok{left( }\SpecialCharTok{\textbackslash{}\textbackslash{}}\StringTok{boldsymbol\{}\SpecialCharTok{\textbackslash{}\textbackslash{}}\StringTok{theta\} }\SpecialCharTok{\textbackslash{}\textbackslash{}}\StringTok{right)$"}\NormalTok{,}
  \AttributeTok{escape =} \ConstantTok{FALSE}
\NormalTok{)}
\end{Highlighting}
\end{Shaded}

\begin{table}

\caption{\label{tab:unnamed-chunk-4}$\boldsymbol{\Sigma} \left( \boldsymbol{\theta} \right)$}
\centering
\begin{tabular}[t]{l|r|r|r}
\hline
  & $v_1$ & $v_2$ & $v_3$\\
\hline
$v_1$ & 59.180667 & 9.076802 & 47.65985\\
\hline
$v_2$ & 9.076802 & 7.151261 & 0.00000\\
\hline
$v_3$ & 47.659854 & 0.000000 & 47.65985\\
\hline
\end{tabular}
\end{table}

\hypertarget{specification-2---observed-variables}{%
\section{Specification 2 - Observed Variables}\label{specification-2---observed-variables}}

\hypertarget{matrix-notation-1}{%
\subsection{Matrix Notation}\label{matrix-notation-1}}

\begin{equation}
  \mathbf{v}
  =
  \begin{bmatrix}
    v_1 \\
    v_2
  \end{bmatrix}
\end{equation}

\begin{equation}
  \begin{split}
    \mathbf{A}
    &=
    \begin{bmatrix}
      0 & a_{1, 2} \\
      0 & 0
    \end{bmatrix} \\
    &=
    \begin{bmatrix}
      0 & 1.269259 \\
      0 & 0
    \end{bmatrix}
  \end{split}
\end{equation}

\begin{equation}
  \begin{split}
    \boldsymbol{\Omega}
    &=
    \begin{bmatrix}
      \omega_{1, 1} & 0             \\
      0             & \omega_{2, 2}
    \end{bmatrix} \\
    &=
    \begin{bmatrix}
      47.659854 & 0 \\
      0 & 7.151261
    \end{bmatrix}
  \end{split}
\end{equation}

\begin{equation}
  \begin{split}
    \mathbf{m}
    &=
    \begin{bmatrix}
      m_1 \\
      m_2
    \end{bmatrix} \\
    &=
    \begin{bmatrix}
      -3.951208 \\
      13.038328
    \end{bmatrix}
  \end{split}
\end{equation}

\begin{equation}
  \mathbf{F}
  =
  \begin{bmatrix}
    1 & 0 \\
    0 & 1
  \end{bmatrix}
\end{equation}

\begin{figure}
\centering
\includegraphics{ram-notes_files/figure-latex/regression-02-1.png}
\caption{\label{fig:regression-02}The Simple Linear Regression Model (without \(v_3\))}
\end{figure}

\hypertarget{using-the-ram-package-1}{%
\subsubsection{\texorpdfstring{Using the \texttt{ram()} Package}{Using the ram() Package}}\label{using-the-ram-package-1}}

\begin{Shaded}
\begin{Highlighting}[]
\NormalTok{knitr}\SpecialCharTok{::}\FunctionTok{kable}\NormalTok{(}
\NormalTok{  ram}\SpecialCharTok{::}\FunctionTok{mutheta}\NormalTok{(}
\NormalTok{    m,}
    \AttributeTok{A =}\NormalTok{ A,}
    \AttributeTok{filter =}\NormalTok{ filter}
\NormalTok{  ),}
  \AttributeTok{col.names =} \StringTok{"$}\SpecialCharTok{\textbackslash{}\textbackslash{}}\StringTok{boldsymbol\{}\SpecialCharTok{\textbackslash{}\textbackslash{}}\StringTok{mu\}$"}\NormalTok{,}
  \AttributeTok{caption =} \StringTok{"$}\SpecialCharTok{\textbackslash{}\textbackslash{}}\StringTok{boldsymbol\{}\SpecialCharTok{\textbackslash{}\textbackslash{}}\StringTok{mu\} }\SpecialCharTok{\textbackslash{}\textbackslash{}}\StringTok{left( }\SpecialCharTok{\textbackslash{}\textbackslash{}}\StringTok{boldsymbol\{}\SpecialCharTok{\textbackslash{}\textbackslash{}}\StringTok{theta\} }\SpecialCharTok{\textbackslash{}\textbackslash{}}\StringTok{right)$"}\NormalTok{,}
  \AttributeTok{escape =} \ConstantTok{FALSE}
\NormalTok{)}
\end{Highlighting}
\end{Shaded}

\begin{table}

\caption{\label{tab:unnamed-chunk-6}$\boldsymbol{\mu} \left( \boldsymbol{\theta} \right)$}
\centering
\begin{tabular}[t]{l|r}
\hline
  & $\boldsymbol{\mu}$\\
\hline
$v_1$ & 12.59781\\
\hline
$v_2$ & 13.03833\\
\hline
\end{tabular}
\end{table}

\begin{Shaded}
\begin{Highlighting}[]
\NormalTok{knitr}\SpecialCharTok{::}\FunctionTok{kable}\NormalTok{(}
\NormalTok{  ram}\SpecialCharTok{::}\FunctionTok{Sigmatheta}\NormalTok{(}
    \AttributeTok{A =}\NormalTok{ A,}
    \AttributeTok{Omega =}\NormalTok{ Omega,}
    \AttributeTok{filter =}\NormalTok{ filter}
\NormalTok{  ),}
  \AttributeTok{caption =} \StringTok{"$}\SpecialCharTok{\textbackslash{}\textbackslash{}}\StringTok{boldsymbol\{}\SpecialCharTok{\textbackslash{}\textbackslash{}}\StringTok{Sigma\} }\SpecialCharTok{\textbackslash{}\textbackslash{}}\StringTok{left( }\SpecialCharTok{\textbackslash{}\textbackslash{}}\StringTok{boldsymbol\{}\SpecialCharTok{\textbackslash{}\textbackslash{}}\StringTok{theta\} }\SpecialCharTok{\textbackslash{}\textbackslash{}}\StringTok{right)$"}\NormalTok{,}
  \AttributeTok{escape =} \ConstantTok{FALSE}
\NormalTok{)}
\end{Highlighting}
\end{Shaded}

\begin{table}

\caption{\label{tab:unnamed-chunk-7}$\boldsymbol{\Sigma} \left( \boldsymbol{\theta} \right)$}
\centering
\begin{tabular}[t]{l|r|r}
\hline
  & $v_1$ & $v_2$\\
\hline
$v_1$ & 59.180667 & 9.076802\\
\hline
$v_2$ & 9.076802 & 7.151261\\
\hline
\end{tabular}
\end{table}

  \bibliography{book.bib,packages.bib}

\end{document}
